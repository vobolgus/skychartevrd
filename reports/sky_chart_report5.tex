\documentclass{SAS-class-skygen}
    
    \newcommand{\hwnum}{Скайчарт от Свята}
	\newcommand{\subject}{№5}
	\newcommand{\skykey}{-32.57 18:12}
    
    \begin{document}
    
    
    
	\begin{center}
		\large\textbf{Ежедневный скайчарт №5}
	\end{center}

	\begin{enumerate}
		\item Обозначьте точку зенита символом \boldsans{Z} и стороны света как \boldsans{N}, \boldsans{E}, \boldsans{S}, \boldsans{W}.
		\item Обозначьте полюс мира символом \boldsans{P}.
		\item Обозначьте точку весны символом \Aries. Или же точку осени символом \Libra.
		\item Проведите большие круги небесного экватора и эклиптики.
		\item Рассчитайте звёздное время скайчарта: \rule{2cm}{0.4pt}
		\item Определите широту места съёмки: \rule{2cm}{0.4pt}
		\item Проведите контуры всех видимых созвездий, а также напишите их обозначения по Байеру.
		\item Отметьте на скайчарте небесные объекты приведённые в таблице ниже.
	\end{enumerate}
	
    \vspace{0.5cm}

    \begin{table}[h!]
    \centering
    \begin{tabular}{ccc}
    \multicolumn{3}{c}{\boldsans{Звёзды}} \\ Садалмелик & Мимоза & Альтаир \\
Вега & Меридиана & Шаула \\
Арктур & Ахернар & Альфа Центавра \\

\end{tabular}
    \hfill
    \begin{tabular}{ccc}
    \multicolumn{3}{c}{\boldsans{Объекты Мессье}} \\ M72 & M73 & M69 \\
M70 & M14 & M5 \\
M107 & M71 & M83 \\

\end{tabular}
    \end{table}
	
	\vspace{0.5cm}
    \begin{center}
    \includegraphics[width=\textwidth]{./pics/sky_chart5.png}
    \end{center}
    
    \end{document}
    